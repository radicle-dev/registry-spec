\newpage
\appendix

% Don't show appendix in TOC.
\addtocontents{toc}{\protect\setcounter{tocdepth}{0}}

\section{Alternative contribution model}
It's possible to drastically reduce the size of checkpoints by merkleizing the commit
history and making contribution claiming an explicit action on the part of
the contributor.

First, we change the checkpoint transaction's $\field{K}{contribs}$ argument from a \emph{list}
of commits to a single \emph{hash}. This hash is the Merkle root of the commit list
that was $\field{K}{contribs}$, with $\tuple{\field{C}{hash}, \field{C}{author}, \field{C}{sig}}$ at each leaf:
% TODO: Is `C_{author}` actually required?
\[
    \field{K}{contribs} = \operatorname{\mathsf{hash}}(
        [\tuple{\field{C}{hash}, \field{C}{author}, \field{C}{sig}}]
    ),
\]
where \textsf{hash} is a Merkle hash function. $\field{K}{contribs}$ is
now constant-sized. Note that in this alternative model, $\field{C}{parent}$ is
no longer used, since it can't be verified.

Then, we ask contributors to claim contributions if they wish to receive rewards, with
a new \textsf{claim-contribution} transaction:
\[
    \tx{claim-contribution}{\field{P}{id}, C_1 \dots C_n, \field{C}{proof}}
\]

\spec{
    \inputs
    \begin{itemize}
        \item $\field{P}{id}$ is the id of the project $P$ to claim the
            contributions on.
        \item $C_1 \dots C_n$ is the list of commits to claim, where $C =
            \tuple{\field{C}{hash}, \field{C}{sig}}$, and $\field{C}{author} =
            \alpha$, the author of the transaction. Note that $\field{C}{author}$
            is not actually included in the transaction, since it is assumed to always
            be $\alpha$.
        \item $\field{C}{proof}$ is the Merkle proof asserting that $C_1 \dots C_n$
            are included in the $\field{K}{contribs}$ set. It is expected that
            $\field{C}{proof}$ can use structural sharing to compress the Merkle paths
            such that the transaction does not grow linearly in $n$. Formally, each
            $\field{C}{proof}$ is an ordered set $\{\pi_1 \dots \pi_n\}$ of Merkle
            paths associated with $C_1 \dots C_n$.
    \end{itemize}
    \validation
    \begin{itemize}
        \item $\field{C}{proof}$ is valid if each path $\pi$ hashes to the same
            root hash $r$, and there is a checkpoint $K$ in $\field{P}{k}$'s
            ancestry such that $\field{K}{contribs} = r$.
        \item For each $C \in C_1 \dots C_n$, $\field{C}{sig}$ is a signature by the
            author $\alpha$.
        \item $C_1 \dots C_n$ have never been claimed before on $P$. Note that if
            two projects have a shared ancestry, it's possible to claim
            contributions once for \emph{each} project.
        % TODO: Explain how the proof is constructed?
    \end{itemize}
    \outputs
    \begin{itemize}
        \item $\alpha_{bal'} = \alpha_{bal} + v$, where $\alpha_{bal'}$ is the
            author's new account balance, and $v = \field{P}{contract}(t)$ where
            $t$ is the \textsf{claim-contribution} transaction.
    \end{itemize}
}

\section{Projects}

\subsection{Project proposal}
The act of proposing a project for registration under a unique name and domain:
\[
    \tx{propose-project}{\field{P}{id}, \field{P}{k}, \field{P}{contract},
        \field{P}{proof}, \field{P}{meta}}
\]
This transaction requires a deposit $\deposit{propose-project}$.

\spec{
\inputs
\begin{itemize}
    \item $\field{P}{id} = \tuple{\field{P}{name}, \field{P}{org}}$
    \item $\field{P}{name}$ is the unique name being requested, where
    \item $\field{P}{domain}$ is the domain under which $\field{P}{name}$ is
        being registered, which together form the unique identifier
        $\field{P}{id}$,
    \item $\field{P}{k}$ is the id of the initial \emph{checkpoint} associated
        with this project, formally $k_0$. This checkpoint must always remain
        in the project ancestry,
    \item $\field{P}{contract}$ is the initial project contract that includes
        the initial permission set around the project,
    \item $\field{P}{proof}$ is a byte array of up to 4096 bytes supplied to
        prove the legitimacy of this project proposal---it is verified
        \emph{off-registry} during project approval (See \S\ref{s:accepting-projects}),
    \item $\field{P}{meta}$ is a dictionary of metadata to be associated with
        $P$. For example, identities on other platforms. Note that once submitted,
        the metadata is immutable.
\end{itemize}
\validation
\begin{itemize}
    \item $\field{P}{name}$ must be unique, \ie not currently registered under
        $\field{P}{domain}$, between $1$ and $32$ characters long, and valid
        \textsf{UTF-8}---note that it is perfectly valid for multiple proposals
        to co-exist with the same $\field{P}{id}$.
    \item $\field{P}{domain}$ must be an existing domain,
    \item $\field{P}{proof}$'s size is $\leq 4096$ bytes,
    \item $\field{P}{k}$ must represent an existing checkpoint,
    \item $\alpha_{bal} \geq \deposit{propose-project}$.
\end{itemize}
\outputs
\begin{itemize}
    \item $P \in \pending$, where $\pending$ is the set of proposals that
        are in a ``pending'' state, waiting to be accepted or rejected (See
        \S\ref{s:accepting-projects}). Note that $\pending$ and $\projects$
        are disjoint,
    \item $\alpha_{bal'} = \alpha_{bal} - \deposit{propose-project}$.
\end{itemize}}

\subsection{Accepting and rejecting a project}
\label{s:accepting-projects}
The act of accepting or rejecting a project being registered:
\[
    \tx{accept-project}{\field{t}{hash}}
\]
or
\[
    \tx{reject-project}{\field{t}{hash}}
\]

\inputs
\begin{itemize}
    \item $\field{t}{hash}$ is the \emph{transaction hash} of the
        \textsf{propose-project} transaction $t$ of a project $P$
        being accepted or rejected.
\end{itemize}
\validation
\begin{itemize}
    \item The transaction \emph{origin} is a member of $\root$,
    \item $\field{t}{hash}$ must be the hash of an existing transaction of
        type \textsf{propose-project}. In other words, $P \in \pending$,
        where $P$ is the project being registered.
\end{itemize}

\bigskip
\noindent For \textsf{accept-project},
\newline\outputs
\begin{itemize}
    \item $P \in \projects$
    \item $P \notin \pending$
    \item $\field{P}{account} = \tuple{\field{A}{id}, 0, 0}$
\end{itemize}
\bigskip
\noindent For \textsf{reject-project},
\newline\outputs
\begin{itemize}
    \item $P \notin \projects$
    \item $P \notin \pending$
\end{itemize}

\subsection{Project proposal withdrawal}
It's possible to withdraw a project from proposal if it hasn't been
accepted or rejected yet with:
\[
    \tx{withdraw-project}{\field{t}{hash}}.
\]
This in turn returns the deposit made on \textsf{propose-project}.

\spec{
    \inputs
    \begin{itemize}
        \item $\field{t}{hash}$ is the transaction hash of the \textsf{propose-project}
            transaction $t$ for some project $P$ that should be withdrawn.
    \end{itemize}
    \validation
    \begin{itemize}
        \item $P \in \pending$
    \end{itemize}
    \outputs
    \begin{itemize}
        \item $P \notin \pending$
        \item $\alpha_{bal'} = \alpha_{bal} + \deposit{propose-project}$.
    \end{itemize}
}

\subsection{Project Tokens}
Projects are able to issue non-fungible tokens (NFTs) that can be used to
access certain services or features around the project, such as support,
licenses, issue and feature prioritization, or access to premium content.

With NFTs, the registry functions as a decentralized ``licensing server'',
which grants a token holder access to certain services. Acquiring NFTs
is done by transfer from another holder, or purchase directly from the
project.

A token is formally represented as:
\[
    \theta = \tuple{\field{\theta}{id}, \field{\theta}{proj}, \field{\theta}{tag},
                    \field{\theta}{price}}
\]
where $\field{\theta}{id}$ uniquely identifies a token unit. The set of
all tokens is $\Theta$.

\subsection{Token issuance}
\[
    \tx{issue-tokens}{\field{\theta}{proj}, \field{\theta}{tag}, \field{\theta}{price}, n}
\]

\spec{
    \inputs
    \begin{itemize}
        \item $\field{\theta}{proj}$ is the id of the project $P$ issuing the token,
        \item $\field{\theta}{tag}$ is a project-scoped tag to be associated with the issued tokens,
        \item $\field{\theta}{price}$ is the price for one token, in the native currency,
        \item $n$ is the number of tokens to issue with that tag and price.
    \end{itemize}
    \validation
    \begin{itemize}
        \item $P \in \projects$,
        \item $\field{\theta}{tag} \in \bin_{32}$,
        \item $\field{\theta}{price} \in \nat_{\geq 0}$,
        \item $n \geq 1$,
    \end{itemize}
    \outputs
    \begin{itemize}
        \item $\Theta' = \Theta \cup \theta_{0} \ldots \theta_{n}$
    \end{itemize}}
Note that $\field{\theta}{tag}$ can be re-used to re-issue tokens with the same tag (but a potentially different price) multiple times. The tag is an arbitrary value meant to represent
the type of NFT being purchased.

\subsection{Token revocation}
\[
    \tx{revoke-tokens}{\field{\theta}{id_0} \ldots \field{\theta}{id_n}}
\]

\subsection{Token purchase}
\[
    \tx{purchase-token}{\field{\theta}{id}}
\]

\subsection{Token transfer}
\[
    \tx{transfer-token}{\field{\theta}{id}, \field{A}{id}}
\]

\section{Root accounts}
Some accounts are considered \emph{privileged}. These `root' accounts,
formally $\root \subset \accounts$ are authorized to conduct certain transactions
that are only valid when originating from these accounts.

The set of accounts in $\root$ is defined at \emph{genesis}, and may not be
further modified in the initial protocol.

\section{Names}

\subsection{Registering a domain}
The act of registering a top-level domain:
\[
    \tx{register-domain}{domain}
\]
\inputs
\begin{itemize}
    \item $domain$ is the unique domain being registered.
\end{itemize}
\validation
\begin{itemize}
    \item The transaction origin $\alpha$ is a member of $\root$,
    \item $domain$ must be available for registration, between $1$ and $32$
        characters long, and valid \textsf{UTF-8}.
\end{itemize}
For example,
\[
    \tx{register-domain}{\mathtt{crates}}
\]

\subsection{Distribution contracts}
A distribution contract can be described as a function:
\[
    f : \txs \to \mathcal{O}
\]
where $\txs$ is any known transaction, and $\mathcal{O}$ is the
function's codomain described by:
\[
    \{\top, \bot\} \cup [\distribution] \cup \nat.
\]
An output in $\{\top, \bot\}$ is reserved for \emph{boolean} contracts,
where $\top$ signifies a transaction $t \in \txs$ is
\emph{authorized} to execute by the contract, and $\bot$ means it
is \emph{unauthorized}.  Note that a transaction can be verified
and included in the transaction ledger $\txs$ yet still be
unauthorized to run.

An output in $[\distribution]$ is reserved for transactions
that should trigger a distribution of funds to one or more
contract-specified beneficiaries, where $\field{A}{id}$ is the
account of one of the beneficiaries, and $\nat$ is the value to
transfer.

An output in $\nat$ is reserved for transactions that should
trigger a transfer of funds to an a priori \emph{known} beneficiary
that shouldn't be determined by the contract. For example, the transaction
origin.

\section{Identifying as a contributor}
The act of identifying yourself as a contributor, by linking a public key used
to sign project contributions, to an account in the registry:
\[
    \tx{identify}{\field{I}{pk}, \field{I}{proof}}
\]

\spec{
    \inputs
    \begin{itemize}
        \item $\field{I}{pk}$ is the public key that is to be associated with
            the \emph{origin} account $\alpha$ if this transaction succeeds.
        \item $\field{I}{proof}$ is a proof verifying that the transaction
            author owns $\field{I}{pk}$.
    \end{itemize}
    \validation
    \begin{itemize}
        \item $\field{I}{pk}$ is not already associated with an account,
        \item $\field{I}{proof}$ is $\alpha_{id}$ signed by the secret key $sk$ that
            $\field{I}{pk}$ was derived from. In other words,
            \[
                \field{I}{proof} = \mathsf{encrypt(hash(\alpha_{id}), \mathit{sk})}
            \]
            which is valid if
            \[
                \mathsf{decrypt}(\field{I}{proof}, \field{I}{pk}) \equiv
                    \mathsf{hash(\alpha_{id})}
            \]
    \end{itemize}}

\subsection{Forgetting identities}
When a public key associated with the \textsf{identify} transaction is lost or no
longer used, the following transaction will `forget' the association:
\[
    \tx{forget}{\field{I}{pk}}
\]
