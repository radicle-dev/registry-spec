\section{Transactions}
\label{s:transactions}

All transactions on the registry take the form $\tx{transaction}{arg_1, \ldots,
arg_n}$, where $arg_1, \ldots arg_n$ are the \emph{inputs} and $\sigma$ is the
\textsf{EdDSA} signature of the author of the transaction. Transactions always
have an \emph{author} and an \emph{origin} (formally $\alpha$), which is the
author's account.

Transactions can be uniquely identified by their \emph{hash}. The set of all
\emph{known} transactions is $\txs$ (the ``ledger''), and the set of all known
transaction hashes is $\field{\txs}{hash}$.

\section{Accounts}
An account $A$ is a tuple:
\[
    A = \tuple{\field{A}{id}, \field{A}{nonce}, \field{A}{bal}}
\]

\spec{
    \definition
    \begin{itemize}
        \item $\field{A}{id}$ is the unique account identifier obtained by hashing
            the account owner's public key,
        \item $\field{A}{nonce}$ is a number which starts at $0$ and is incremented
            every time a transaction originates from this account.
        \item $\field{A}{bal}$ is the account's balance in the smallest denomination,
            and $\field{A}{bal} \in \nat_{\geq 0}$.
    \end{itemize}}
The set of all accounts is $\accounts$. Accounts are never created or destroyed,
rather, if they have never been used to transact, they have an initial state of:
\[
    A = \tuple{\field{A}{id}, 0, 0}
\]
Hence, for all valid account ids, there exists an account with that id. In other
words, $\forall a \in \field{A}{id} (A \in \accounts)$.

Note that accounts can \emph{never be removed}, since that would violate the
invariant that nonces are only ever incremented, and removing an account is
equivalent to setting $\field{A}{nonce}$ and $\field{A}{bal}$ to $0$.

\subsection{Transferring value}
The act of transferring coins between two accounts:
\[
    \tx{transfer}{\field{A}{id}, v}
\]
which will transfer value from the transaction origin $\origin$ to account $A$.

\spec{
    \inputs
    \begin{itemize}
        \item $\field{A}{id}$ is the account id of the \emph{receiver} of the transfer,
        \item $v$ is the value or `balance' to transfer from the origin to the receiver,
            in the smallest denomination.
    \end{itemize}

    \validation
    \begin{itemize}
        \item The transfer balance is positive, or $v \geq 1$,
        \item The origin's balance minus any transaction fee is $\geq v$.
    \end{itemize}
    \outputs
    \begin{itemize}
        \item $v$ is debited from the origin and credited to $A$.
    \end{itemize}
}

\section{Orgs}
An org is a logical grouping of people and projects with common governance
and funds. An org $O$ is a tuple:
\[
    O = \tuple{\field{O}{id}, \field{O}{account},
        \field{O}{members}, \field{O}{projs}, \field{O}{contract}}
\]

\spec{
    \definition
    \begin{itemize}
        \item $\field{O}{id}$ is the globally unique org identifier,
        \item $\field{O}{account}$ is the org account or \emph{fund},
        \item $\field{O}{members}$ is the set of registered org members,
        \item $\field{O}{projs}$ is the set of registered projects under this org,
        \item $\field{O}{contract}$ is the org contract, which governs
            permissions around the org, as well as its fund. It can be
            described as a function:
            \[
                f : \txs \to \{\top, \bot\}
            \]
            where $\txs$ is any transaction $t$ operating on an org, and $\top$
            signifies $t$ is \emph{authorized} to execute by the contract, while
            $\bot$ means it is \emph{unauthorized}.  Note that a transaction
            can be verified and included in the transaction ledger $\txs$ yet
            still be unauthorized to run by the contract
    \end{itemize}}

\subsection{Registering}
\[
    \tx{register-org}{\field{O}{id}, \field{O}{contract}}
\]

\spec{
\inputs
\begin{itemize}
    \item $\field{O}{id}$ is the unique identifier being registered,
    \item $\field{O}{contract}$ is the initial org contract that includes
        the initial permission set around the org.
\end{itemize}
\validation
\begin{itemize}
    \item $\field{O}{id}$ must be unique, \ie not currently in use as an org
        identifier {\em nor} as a user identifier, and comply with the following
        rules:
        \begin{itemize}
            \item Must be between $1$ and $32$ bytes long.
            \item Must only contain \texttt{a-z}, \texttt{0-9} and `\texttt{-}'
                characters.
            \item Must not start or end with a `\texttt{-}'.
            \item Must not contain sequences of more than one `\texttt{-}'.
        \end{itemize}
    \item $\field{\alpha}{bal} \geq \deposit{register-org}$.
\end{itemize}
\outputs
\begin{itemize}
    \item $O \in \orgs$, where $O =
        \tuple{\field{O}{id},
            \field{O}{account},
            \{\field{\origin}{id}\},
            \varnothing,
            \field{O}{contract}}$
    \item $\field{O}{account} \in \accounts$,
    \item $\field{\alpha}{bal'} = \field{\alpha}{bal} - \deposit{register-org}$.
\end{itemize}}

\subsection{Unregistering}
\[
    \tx{unregister-org}{\field{O}{id}}
\]

\spec{
\inputs
\begin{itemize}
    \item $\field{O}{id}$ is the identifier of the org being unregistered,
\end{itemize}
\validation
\begin{itemize}
    \item $O \in \orgs$,
    \item $\field{O}{members} = \{\field{\origin}{id}\}$, the transaction origin must be the
        only member,
    \item $\field{O}{projs} = \varnothing$, there must be no projects under the
        org,
\end{itemize}
\outputs
\begin{itemize}
    \item $O \notin \orgs$,
    \item $\field{\alpha}{bal'} = \field{\alpha}{bal} + \deposit{register-org} + \field{O}{account_{bal}}$.
\end{itemize}}

\subsection{Registering members}
\[
    \tx{register-member}{\field{O}{id}, \field{A}{id}}
\]

\spec{
\inputs
\begin{itemize}
    \item $\field{A}{id}$ is the account id being registered as a member,
    \item $\field{O}{id}$ is the id of the org under which to register $A$,
\end{itemize}
\validation
\begin{itemize}
    \item $O \in \orgs$,
    \item $\field{A}{id}$ must not already be registered under $O$, or $\field{A}{id} \notin \field{O}{members}$
    \item The transaction author is authorized to execute \txname{register-member},
    \item $\alpha_{bal} \geq \deposit{register-member}$.
\end{itemize}
\outputs
\begin{itemize}
    \item $\field{A}{id} \in \field{O}{members}$,
    \item $\alpha_{bal'} = \alpha_{bal} - \deposit{register-member}$.
\end{itemize}}

\subsection{Unregistering members}
\[
    \tx{unregister-member}{\field{O}{id}, \field{A}{id}}
\]

\spec{
\inputs
\begin{itemize}
    \item $\field{O}{id}$ is the id of the org under which the member is registered,
    \item $\field{A}{id}$ is the account id of the member being unregistered,
\end{itemize}
\validation
\begin{itemize}
    \item $O \in \orgs$,
    \item $\field{A}{id} \in \field{O}{members}$,
    \item The transaction author is authorized to execute \txname{unregister-member}.
\end{itemize}
\outputs
\begin{itemize}
    \item $\field{A}{id} \notin \field{O}{members}$,
    \item $\alpha_{bal'} = \alpha_{bal} + \deposit{register-member}$.
\end{itemize}}

\subsection{The Contract}
Every org $O$ in the registry has a contract denoted $\field{O}{contract}$.
The way this contract is invoked is through transactions that act on $O$. For
example, the \textsf{fund} transaction (\S\ref{s:fund}) which transfers value
out of a org is always validated by the org contract before it
is authorized to execute.

A contract is made of a set of \emph{rules} that each handle a specific action
relating to the org. In the \textsf{fund} example, the \textsf{fund}
\emph{rule} would be invoked to determine the outcome of the transaction.

Setting the org's contract is done with:
\[
    \tx{set-contract}{\field{O}{id}, c}
\]

\spec{
    \inputs
    \begin{itemize}
        \item $\field{O}{id}$ is the id of the org,
        \item $c$ is the new contract.
    \end{itemize}
    \validation
    \begin{itemize}
        \item $O \in \orgs$,
        \item The transaction author is authorized to execute \txname{set-contract},
    \end{itemize}
    \outputs
    \begin{itemize}
        \item $\field{O}{contract} = c$
    \end{itemize}}

\subsection{The Fund}
\label{s:fund}
Each org has an associated account $\field{O}{account}$ called the
\emph{fund}. To use that account to fund maintenance of projects,
the \textsf{fund} transaction is used:
\[
    \tx{fund}{\field{O}{id}, \field{A}{id}, v}
\]

\spec{
    \inputs
    \begin{itemize}
        \item $\field{O}{id}$ is the id of the org from which the transfer
            should be initiated,
        \item $\field{A}{id}$ is the id of the account that should receive
            the transfer,
        \item $v$ is the value to transfer.
    \end{itemize}
    \validation
    \begin{itemize}
        \item $\field{O}{account_{bal}} - \deposit{register-org} \geq v$,
        \item The transaction author is authorized to execute \txname{fund},
    \end{itemize}
    \outputs
    \begin{itemize}
        \item $\field{A}{bal'} = \field{A}{bal} + v$
        \item $\field{O}{account_{bal'}} = \field{O}{account_{bal}} - v$
    \end{itemize}}

%%%%%%%%%%%%%%%%%%%%%%%%%%%%%%%%%%%%%%%%%%%%%%%%%%%%%%%%%%%%%%%%%%%%%%%%%%%%%%%%

\section{Projects}
A project $P$ is a tuple:
\[
    P = \tuple{\field{P}{id}, \field{P}{org}, \field{P}{k}, \field{P}{meta}}
\]

\spec{
    \definition
    \begin{itemize}
        \item $\field{P}{id}$ is the unique project identifier within the context of an $\field{P}{org}$,
        \item $\field{P}{org}$ is the org under which this project lives,
        \item $\field{P}{k}$ is the current project \emph{checkpoint} (See \S\ref{s:checkpoint}),
        \item $\field{P}{meta}$ is opaque metadata to associate with $P$. For
            example, the \radicle{} \emph{project id}. Note that once defined, the
            metadata is immutable.
    \end{itemize}}
Projects are registered with the \textsf{register-project} transaction and unregistered
with the \textsf{unregsiter-project} transaction. Projects always exist within the
context of an org.

\subsection{Registering projects}
\[
    \tx{register-project}{\field{P}{org}, \field{P}{id}, \field{P}{k}, \field{P}{meta}}
\]

\spec{
\inputs
\begin{itemize}
    \item $\field{P}{id}$ is the unique project id being requested,
    \item $\field{P}{org}$ is the id of the org under which to register $P$,
    \item $\field{P}{k}$ is the id of the initial \emph{checkpoint} associated
        with this project, formally $k_0$. This checkpoint must always remain
        in the project ancestry,
    \item $\field{P}{meta}$ is associated project metadata.
\end{itemize}
\validation
\begin{itemize}
    \item $\field{P}{org}$ identifies an existing org $O$,
    \item $\field{P}{id}$ is unique under $O$, and complies with the
        following rules:
        \begin{itemize}
            \item Must be between $1$ and $32$ bytes long.
            \item Must only contain \texttt{a-z}, \texttt{0-9}, `\texttt{-}'
                and `\texttt{\_}' characters.
        \end{itemize}
    \item $\field{P}{k}$ represents an existing checkpoint,
    \item $\field{P}{meta}$ is $\leq 128$ bytes long,
    \item The transaction author is authorized to execute \txname{register-project},
    \item $\alpha_{bal} \geq \deposit{register-project}$.
\end{itemize}
\outputs
\begin{itemize}
    \item $P \in \field{O}{projs}$, where $\field{O}{id} = \field{P}{org}$,
    \item $\alpha_{bal'} = \alpha_{bal} - \deposit{register-project}$.
\end{itemize}}

\subsection{Unregistering}
\[
    \tx{unregister-project}{\field{P}{org}, \field{P}{id}}
\]

\spec{
\inputs
\begin{itemize}
    \item $\field{P}{id}$ is the project id being unregistered,
    \item $\field{P}{org}$ is the id of the org under which $P$ lives,
\end{itemize}
\validation
\begin{itemize}
    \item $\field{P}{org}$ identifies an existing org $O$,
    \item $P \in \field{O}{projs}$,
    \item The transaction author is authorized to execute \txname{unregister-project}.
\end{itemize}
\outputs
\begin{itemize}
    \item $P \notin \field{O}{projs}$,
    % TODO: Does the deposit go back to the org account instead?
    \item $\alpha_{bal'} = \alpha_{bal} + \deposit{register-project}$.
\end{itemize}}


\subsection{Creating a checkpoint}
\label{s:checkpoint}
The act of anchoring a project's state in the registry:
\[
    \tx{checkpoint}{\field{K}{parent}, \field{K}{hash}}
\]
Checkpoints within the scope of a single project form a chain going from the
latest, or ``current'' checkpoint $k_{n-1}$ to the first and original
checkpoint $k_0$. Checkpoints are identified by their transaction hash,
so $k \in \field{T}{hash}$.

From the perspective of $k_0$, we can talk of a checkpoint \emph{tree}, since
due to their nature, they are able to represent branching. Hence, the original
checkpoint $k_0$ is also called the \emph{root} checkpoint.

\inputs
\begin{itemize}
    \item $\field{K}{parent}$ is the \emph{id} of the previous or `parent' checkpoint,
    \item $\field{K}{hash}$ is the new hash of the project state,
\end{itemize}
\validation
\begin{itemize}
    \item{$\field{K}{parent}$ refers to an existing checkpoint in the registry,
        or is $\varnothing$.}
    \item{$\field{K}{hash}$ is a valid hash that hasn't been used in a parent
        checkpoint.}
\end{itemize}

\subsection{Setting a checkpoint}
The act of updating the project to point to a new checkpoint:
\[
    \tx{set-checkpoint}{\field{P}{org}, \field{P}{id}, k'}
\]
which updates $\field{P}{checkpoint}$ from $k$ to $k'$.

\spec{
    \inputs
    \begin{itemize}
        \item $\field{P}{org}$ is the id of the org $O$ under which the project lives,
        \item $\field{P}{id}$ is the id of the project being updated,
        \item $k'$ is the id of the checkpoint the project
            should be associated to.
    \end{itemize}
    \validation
    \begin{itemize}
        \item $\field{P}{id} \in \field{O}{projs}$, or $P$ lives under $O$,
        \item $k'$ is a checkpoint which has the original project checkpoint
            $k_0$ in its ancestry,
        \item The transaction author is authorized to execute \textsf{set-checkpoint}.
    \end{itemize}
    \outputs
    \begin{itemize}
        \item $\field{P}{k} = k'$
    \end{itemize}}
Note that the semantics of this transaction allows for projects to revert to
a previous checkpoint, or to adopt a ``fork'', as long as the new checkpoint
shares part of its ancestry with the previous checkpoint.

\section{User Identity}

Identity in the registry serves as a way for users to consolidate the various
keys and external identities they use under a short, human-readable name.

A user $U$ is a logical grouping of {\em identities}, or user identifiers under a
single, unique identifier, $\field{U}{id}$. The set of all users is $\users$.
\[
    U = \tuple{\field{U}{id}, \field{U}{account}, \field{U}{keys}}
\]
\spec{
    \definition
    \begin{itemize}
        \item $\field{U}{id}$ is the globally unique human-readable identifier of the user,
        \item $\field{U}{account}$ is the account id which owns this user identity,
        \item $\field{U}{keys}$ is the set of off-registry public keys associated
            with this identity.
    \end{itemize}}

\subsection{Registering}
We register a new user identity and thus user
\[
    \tx{register-identity}{\field{U}{id}}
\]

\spec{
\inputs
\begin{itemize}
    \item $\field{U}{id}$ is the gobally unique user identifier being registered,
\end{itemize}
\validation
\begin{itemize}
    \item $\field{U}{id}$ must be unique, \ie not currently in use as a user
        identifier {\em nor} as an org identifier, and comply with the following rules:
        \begin{itemize}
            \item Must be between $1$ and $32$ bytes long.
            \item Must only contain \texttt{a-z}, \texttt{0-9} and `\texttt{-}'
                characters.
            \item Must not start or end with a `\texttt{-}'.
            \item Must not contain sequences of more than one `\texttt{-}'.
        \end{itemize}
    \item $\field{\alpha}{bal} \geq \deposit{register-identity}$.
\end{itemize}
\outputs
\begin{itemize}
    \item $U \in \users$, where $U = \tuple{\field{U}{id}, \field{\origin}{id}, \varnothing}$
    \item $\field{\alpha}{bal'} = \field{\alpha}{bal} - \deposit{register-identity}$
\end{itemize}}

\subsection{Unregistering}
\[
    \tx{unregister-identity}{\field{U}{id}}
\]

\spec{
\inputs
\begin{itemize}
    \item $\field{U}{id}$ is the identity being unregistered,
\end{itemize}
\validation
\begin{itemize}
    \item $U \in \users$,
    \item $\origin$ must be the owner of this identity, in other words $\field{U}{account} \equiv \field{\origin}{id}$,
\end{itemize}
\outputs
\begin{itemize}
    \item $U \notin \users$,
    \item $\field{\origin}{bal'} = \field{\origin}{bal} + \deposit{register-identity}$.
\end{itemize}}

\subsection{Associating an external key}
The act of associating an external public key to a registered user identity:
\[
    \tx{associate-key}{\field{U}{id}, k, \pi}
\]

\spec{
    \inputs
    \begin{itemize}
        \item $\field{U}{id}$ is the identity under which to associate the key,
        \item $k$ is the public portion of the key pair $\tuple{k, S_k}$ that is to be
            associated,
        \item $\pi$ is a proof or signature verifying that the transaction
            author owns $k$, defined as:
            \[
                \pi = \mathsf{encrypt(hash(\mathit{\field{U}{id}}), \mathit{S_k})}
            \]
            where $S_k$ is the secret key from which $k$ was derived.
    \end{itemize}
    \validation
    \begin{itemize}
        \item $k \notin \field{U}{keys}$,
        \item $k$ is a valid $32$ byte \textsf{Ed25519} key,
        \item $\field{\origin}{id} \equiv \field{U}{account}$,
        \item $\mathsf{decrypt}(\pi, k) \equiv \mathsf{hash(\mathit{\field{U}{id}})}$
    \end{itemize}
    \outputs
    \begin{itemize}
        \item $k \in \field{U}{keys}$
    \end{itemize}}

\subsection{Revoking an external key}
When a public key associated with the \textsf{associate-key} transaction is lost or no
longer used, the following transaction will `revoke' the association:
\[
    \tx{revoke-key}{\field{U}{id}, k}
\]

\spec{
    \inputs
    \begin{itemize}
        \item $\field{U}{id}$ is the identity under which the key is currently associated,
        \item $k$ is the key being revoked,
    \end{itemize}
    \validation
    \begin{itemize}
        \item $k \in \field{U}{keys}$,
        \item $\field{\origin}{id} \equiv \field{U}{account}$,
    \end{itemize}
    \outputs
    \begin{itemize}
        \item $k \notin \field{U}{keys}$
    \end{itemize}}
