\section{Transactions}
\label{s:transactions}

All transactions on the registry take the form $\tx{transaction}{arg_1, \ldots,
arg_n}$, where $arg_1, \ldots arg_n$ are the \emph{inputs} and $\sigma$ is the
\textsf{EdDSA} signature of the author of the transaction. Transactions always
have an \emph{author} and an \emph{origin} (formally $\alpha$), which is the
author's account.

Transactions can be uniquely identified by their \emph{hash}. The set of all
\emph{known} transactions is $\txs$ (the ``ledger''), and the set of all known
transaction hashes is $\field{\txs}{hash}$.

\section{Accounts}
An account $A$ is a tuple:
\[
    A = \tuple{\field{A}{id}, \field{A}{nonce}, \field{A}{bal}}
\]

\spec{
    \definition
    \begin{itemize}
        \item $\field{A}{id}$ is the unique account identifier obtained by hashing
            the account owner's public key,
        \item $\field{A}{nonce}$ is a number which starts at $0$ and is incremented
            every time a transaction originates from this account.
        \item $\field{A}{bal}$ is the account's balance in the smallest denomination,
            and $\field{A}{bal} \in \nat_{\geq 0}$.
    \end{itemize}}
The set of all accounts is $\accounts$. Accounts are never created or destroyed,
rather, if they have never been used to transact, they have an initial state of:
\[
    A = \tuple{\field{A}{id}, 0, 0}
\]
Hence, for all valid account ids, there exists an account with that id. In other
words, $\forall a \in \field{A}{id} (A \in \accounts)$.

Note that accounts can \emph{never be removed}, since that would violate the
invariant that nonces are only ever incremented, and removing an account is
equivalent to setting $\field{A}{nonce}$ and $\field{A}{bal}$ to $0$.

\subsection{Transferring value}
The act of transferring coins between two accounts:
\[
    \tx{transfer}{\field{A}{id}, v}
\]
which will transfer value from the transaction origin $\origin$ to account $A$.

\spec{
    \inputs
    \begin{itemize}
        \item $\field{A}{id}$ is the account id of the \emph{receiver} of the transfer,
        \item $v$ is the value or `balance' to transfer from the origin to the receiver,
            in the smallest denomination.
    \end{itemize}

    \validation
    \begin{itemize}
        \item The transfer balance is positive, or $v \geq 1$,
        \item The origin's balance minus any transaction fee is $\geq v$.
    \end{itemize}
    \outputs
    \begin{itemize}
        \item $v$ is debited from the origin and credited to $A$.
    \end{itemize}
}

\section{Orgs}
An org is a logical grouping of people and projects with common governance
and funds. An org $O$ is a tuple:
\[
    O = \tuple{\field{O}{id}, \field{O}{account},
        \field{O}{members}, \field{O}{projs}, \field{O}{contract}}
\]

\spec{
    \definition
    \begin{itemize}
        \item $\field{O}{id}$ is the globally unique org identifier,
        \item $\field{O}{account}$ is the org account or \emph{fund},
        \item $\field{O}{members}$ is the set of registered org members,
        \item $\field{O}{projs}$ is the set of registered projects under this org,
        \item $\field{O}{contract}$ is the org contract, which governs
            permissions around the org, as well as its fund. It can be
            described as a function:
            \[
                f : \txs \to \{\top, \bot\}
            \]
            where $\txs$ is any transaction $t$ operating on an org, and $\top$
            signifies $t$ is \emph{authorized} to execute by the contract, while
            $\bot$ means it is \emph{unauthorized}.  Note that a transaction
            can be verified and included in the transaction ledger $\txs$ yet
            still be unauthorized to run by the contract
    \end{itemize}}

\subsection{Registering orgs}
\[
    \tx{register-org}{\field{O}{id}, \field{O}{contract}}
\]

\spec{
\inputs
\begin{itemize}
    \item $\field{O}{id}$ is the unique identifier being registered,
    \item $\field{O}{contract}$ is the initial org contract that includes
        the initial permission set around the org.
\end{itemize}
\validation
\begin{itemize}
    \item $\field{O}{id}$ must be unique, \ie not currently in use,
        between $1$ and $32$ bytes long, and valid \textsf{UTF-8}.
    \item $\field{\alpha}{bal} \geq \deposit{register-org}$.
\end{itemize}
\outputs
\begin{itemize}
    \item $O \in \orgs$, where $O =
        \tuple{\field{O}{id},
            \field{O}{account},
            \{\origin\},
            \varnothing,
            \field{O}{contract}}$
    \item $\field{O}{account} \in \accounts$,
    \item $\field{\alpha}{bal'} = \field{\alpha}{bal} - \deposit{register-org}$.
\end{itemize}}

\subsection{Unregistering orgs}
\[
    \tx{unregister-org}{\field{O}{id}}
\]

\spec{
\inputs
\begin{itemize}
    \item $\field{O}{id}$ is the identifier of the org being unregistered,
\end{itemize}
\validation
\begin{itemize}
    \item $O \in \orgs$,
    \item $\field{O}{members} = \{\origin\}$, the transaction origin must be the
        only member,
    \item $\field{O}{projs} = \varnothing$, there must be no projects under the
        org,
\end{itemize}
\outputs
\begin{itemize}
    \item $O \notin \orgs$,
    \item $\field{\alpha}{bal'} = \field{\alpha}{bal} + \deposit{register-org} + \field{O}{account_{bal}}$.
\end{itemize}}

\subsection{Registering org members}
\[
    \tx{register-member}{\field{O}{id}, \field{A}{id}}
\]

\spec{
\inputs
\begin{itemize}
    \item $\field{A}{id}$ is the account id being registered as a member,
    \item $\field{O}{id}$ is the id of the org under which to register $A$,
\end{itemize}
\validation
\begin{itemize}
    \item $O \in \orgs$,
    \item $\field{A}{id}$ must not already be registered under $O$, or $\field{A}{id} \notin \field{O}{members}$
    \item The transaction author is authorized to execute \txname{register-member},
    \item $\alpha_{bal} \geq \deposit{register-member}$.
\end{itemize}
\outputs
\begin{itemize}
    \item $\field{A}{id} \in \field{O}{members}$,
    \item $\alpha_{bal'} = \alpha_{bal} - \deposit{register-member}$.
\end{itemize}}

\subsection{Unregistering org members}
\[
    \tx{unregister-member}{\field{O}{id}, \field{A}{id}}
\]

\spec{
\inputs
\begin{itemize}
    \item $\field{O}{id}$ is the id of the org under which the member is registered,
    \item $\field{A}{id}$ is the account id of the member being unregistered,
\end{itemize}
\validation
\begin{itemize}
    \item $O \in \orgs$,
    \item $\field{A}{id} \in \field{O}{members}$,
    \item The transaction author is authorized to execute \txname{unregister-member}.
\end{itemize}
\outputs
\begin{itemize}
    \item $\field{A}{id} \notin \field{O}{members}$,
    \item $\alpha_{bal'} = \alpha_{bal} + \deposit{register-member}$.
\end{itemize}}

\subsection{The org contract}
Every org $O$ in the registry has a contract denoted $\field{O}{contract}$.
The way this contract is invoked is through transactions that act on $O$. For
example, the \textsf{fund} transaction (\S\ref{s:fund}) which transfers value
out of a org is always validated by the org contract before it
is authorized to execute.

A contract is made of a set of \emph{rules} that each handle a specific action
relating to the org. In the \textsf{fund} example, the \textsf{fund}
\emph{rule} would be invoked to determine the outcome of the transaction.

\subsection{Updating the org contract}
The act of updating a org's contract:
\[
    \tx{set-contract}{\field{O}{id}, c}
\]

\spec{
    \inputs
    \begin{itemize}
        \item $\field{O}{id}$ is the id of the org,
        \item $c$ is the new contract.
    \end{itemize}
    \validation
    \begin{itemize}
        \item $O \in \orgs$,
        \item The transaction author is authorized to execute \txname{set-contract},
    \end{itemize}
    \outputs
    \begin{itemize}
        \item $\field{O}{contract} = c$
    \end{itemize}}

%%%%%%%%%%%%%%%%%%%%%%%%%%%%%%%%%%%%%%%%%%%%%%%%%%%%%%%%%%%%%%%%%%%%%%%%%%%%%%%%

\section{Projects}
A project $P$ is a tuple:
\[
    P = \tuple{\field{P}{id}, \field{P}{org}, \field{P}{k}, \field{P}{proof}, \field{P}{meta}}
\]

\spec{
    \definition
    \begin{itemize}
        \item $\field{P}{id}$ is the unique project identifier within the context of an $\field{P}{org}$,
        \item $\field{P}{org}$ is the org under which this project lives,
        \item $\field{P}{k}$ is the current project \emph{checkpoint} (See \S\ref{s:checkpoint}),
        \item $\field{P}{proof}$ is the proof that was supplied during proposal,
            verifying the owner's authority over the project,
        \item $\field{P}{meta}$ is opaque metadata to associate with $P$. For
            example, the \radicle{} \emph{project id}. Note that once defined, the
            metadata is immutable.
    \end{itemize}}
Projects are registered with the \textsf{register-project} transaction and unregistered
with the \textsf{unregsiter-project} transaction. Projects always exist within the
context of an org.

\subsection{Registering}
\[
    \tx{register-project}{\field{P}{org}, \field{P}{id}, \field{P}{k}, \field{P}{meta}}
\]

\spec{
\inputs
\begin{itemize}
    \item $\field{P}{id}$ is the unique project id being requested,
    \item $\field{P}{org}$ is the id of the org under which to register $P$,
    \item $\field{P}{k}$ is the id of the initial \emph{checkpoint} associated
        with this project, formally $k_0$. This checkpoint must always remain
        in the project ancestry,
    \item $\field{P}{meta}$ is associated project metadata.
\end{itemize}
\validation
\begin{itemize}
    \item $\field{P}{org}$ identifies an existing org $O$,
    \item $\field{P}{id}$ is unique under $O$, between $1$ and $32$
        bytes long, and valid \textsf{UTF-8}.
    \item $\field{P}{k}$ represents an existing checkpoint,
    \item $\field{P}{meta}$ is $\leq 128$ bytes long,
    \item The transaction author is authorized to execute \txname{register-project},
    \item $\alpha_{bal} \geq \deposit{register-project}$.
\end{itemize}
\outputs
\begin{itemize}
    \item $P \in \field{O}{projs}$, where $\field{O}{id} = \field{P}{org}$,
    \item $\alpha_{bal'} = \alpha_{bal} - \deposit{register-project}$.
\end{itemize}}

\subsection{Unregistering}
\[
    \tx{unregister-project}{\field{P}{org}, \field{P}{id}}
\]

\spec{
\inputs
\begin{itemize}
    \item $\field{P}{id}$ is the project id being unregistered,
    \item $\field{P}{org}$ is the id of the org under which $P$ lives,
\end{itemize}
\validation
\begin{itemize}
    \item $\field{P}{org}$ identifies an existing org $O$,
    \item $P \in \field{O}{projs}$,
    \item The transaction author is authorized to execute \txname{unregister-project}.
\end{itemize}
\outputs
\begin{itemize}
    \item $P \notin \field{O}{projs}$,
    % TODO: Does the deposit go back to the org account instead?
    \item $\alpha_{bal'} = \alpha_{bal} + \deposit{register-project}$.
\end{itemize}}


\subsection{Checkpointing}
\label{s:checkpoint}
The act of anchoring a project's state and updating the network graph:
\[
    \tx{checkpoint}{\field{K}{parent}, \field{K}{hash}, \field{K}{version},
    \field{K}{contribs}, \field{K}{deps}}
\]
Checkpoints within the scope of a single project form a chain going from the
latest, or ``current'' checkpoint $k_{n-1}$ to the first and original
checkpoint $k_0$. Checkpoints are identified by their transaction hash,
so $k \in \field{T}{hash}$.

From the perspective of $k_0$, we can talk of a checkpoint \emph{tree}, since
due to their nature, they are able to represent branching. Hence, the original
checkpoint $k_0$ is also called the \emph{root} checkpoint.

\inputs
\begin{itemize}
    \item $\field{K}{parent}$ is the \emph{id} of the previous or `parent' checkpoint,
    \item $\field{K}{hash}$ is the new hash of the project state,
    \item $\field{K}{version}$ is the current version of the project,
    \item $\field{K}{contribs}$ is the list of contributions since $\field{K}{parent}$,
    \item $\field{K}{deps}$ is the list of dependency updates since the $\field{K}{parent}$.
\end{itemize}
\validation
\begin{itemize}
    \item{$\field{K}{parent}$ refers to an existing checkpoint in the registry,
        or is $\varnothing$.}
    \item{$\field{K}{hash}$ is a valid hash that hasn't been used in a parent
        checkpoint.}
    \item{$\field{K}{version}$ is a string between $1$ and $32$ bytes long that
        \emph{may} have been used in a previous project checkpoint.}
    \item{$\field{K}{contribs}$ is a valid contribution list (See \S
        \ref{s:checkpoint-contribs}).}
    \item{$\field{K}{deps}$ is a valid dependency update list (See \S
        \ref{s:checkpoint-deps}).}
\end{itemize}

\subsection{Contributions}
\label{s:checkpoint-contribs}
The list $\field{K}{contribs}$ supplied to the $\mathsf{checkpoint}$
transaction is of the form:

\[
    \field{K}{contribs} = [\tuple{\field{C}{parent}, \field{C}{hash},
    \field{C}{author}, \field{C}{sig}}],
\]

\spec{
    \definition
    \begin{itemize}
        \item $\field{C}{parent}$ is the hash of the parent contribution, or
            $\varnothing$ if this is the first contribution of the first checkpoint
            of the project.
        \item $\field{C}{hash}$ is the hash of the corresponding commit,
        \item $\field{C}{author}$ is the public signing key of the commit referred
            to by $C$,
        \item $\field{C}{sig}$ is the author's \textsf{GPG} signature.
    \end{itemize}
    \validation
    \begin{itemize}
        \item $\field{C}{parent}$ is a valid \textsf{SHA-1} hash or
            $\varnothing$ if this is the first contribution. Note that if $C$
            is $\field{K}{contribs}$'s first item, and $C'$ is the \emph{last}
            item of the \emph{parent} checkpoint's contributions list, then
            $\field{C'}{hash}$ and $\field{C}{parent}$ must be equal, such that
            no gaps between contributions exist.

        \item $\field{C}{hash}$ is a valid \textsf{SHA-1} hash,
        \item $\field{C}{author}$ is the creator of $\field{C}{sig}$,
        \item $\field{C}{sig}$ is a valid signature of $\field{C}{hash}$.
    \end{itemize}
}
Because all changes to a project's source code are described in checkpoints, it
is possible to reconstruct a full hash-linked list of contributions for the
entire project. When cross-referenced with the project's repository, this
constitutes a complete historical record of who authored what code. This
ensures the project history is auditable and tamper-proof, while providing
fundamental information to for the network graph $\netgraph$. Note that only
contribution \emph{metadata} is stored on-chain.

\subsection{Dependency updates}
\label{s:checkpoint-deps}
Conceptually, a project $P$ depends on another project $P'$ if it is an
``input'' to $P$ in some way: $P$ references $P'$ or parts of $P'$ in its
source code, or $P'$ is a build/test dependency.

The dependency update list $\field{P}{deps}$ is a list of \emph{dependency
  updates}, one of:
\[
    \begin{cases}
        \depend(\field{P'}{id}, \field{P'}{version}) \\
        \undepend(\field{P'}{id}, \field{P'}{version})
    \end{cases}
\]
which refer to the project $P'$ at a specific version $\field{P'}{version}$.
The $\depend$ update adds a new dependency while the $\undepend$ update removes
a dependency. The updates are processed in order with $\depend$ only being
valid if it adds a dependency that the project does not already have and
$\undepend$ only being valid for current dependencies. The checkpoint is
invalid if the update list contains duplicates.

% NOTE: Should the checkpoints include the hash of the project being depended
% on? This would add another check for projects registering.

\bigskip
\validation
\begin{itemize}
    \item $\field{P'}{id}$ must be a valid project id, but \emph{does not}
        have to refer to an existing id in the registry. This allows dependent
        projects to checkpoint dependencies that have not yet been registered.
    \item $\field{P'}{version}$ must be a valid version string, but \emph{does not}
        have to refer to an existing version of $P'$. This allows dependent projects
        to checkpoint before their dependencies.
\end{itemize}
As a project maintainer, adding a dependency signals a variety of things
depending of the nature of the project:
\begin{itemize}
\item They have verified that $P$ indeed depends on this specific
  version of $P'$.
\item That $P'$ is suitable as a dependency for $P$, \eg{} if $P$ has
  very high security requirements, that $P'$ fulfills these.
\end{itemize}
\noindent Since contributions to a project carry additional
weight---potentially increasing a project's rank---there is an incentive
for maintainers to checkpoint their projects regularly.  Similarly, adding
dependencies may increase connectivity in the network graph, which may in turn
indirectly improve a project's rank.

\subsection{Setting the project checkpoint}
The act of updating the project to point to a new checkpoint:
\[
    \tx{set-checkpoint}{\field{P}{id}, k'}
\]
which updates $\field{P}{checkpoint}$ from $k$ to $k'$.

\spec{
    \inputs
    \begin{itemize}
        \item $\field{P}{id}$ is the id of the project being updated,
        \item $k'$ is the id of the checkpoint the project
            should be associated to.
    \end{itemize}
    \validation
    \begin{itemize}
        \item $\field{P}{id}$ refers to a project that has been accepted
            in the registry,
        \item $k'$ is a checkpoint which has the original project checkpoint
            $k_0$ in its ancestry.
        \item $\field{P}{contract}(\txmsg{set-checkpoint}{\field{P}{id}, k'}) \equiv \top$
    \end{itemize}
    \outputs
    \begin{itemize}
        \item $\field{P}{k} = k'$
    \end{itemize}}
Note that the semantics of this transaction allows for projects to revert to
a previous checkpoint, or to adopt a ``fork'', as long as the new checkpoint
shares part of its ancestry with the previous checkpoint.

\subsection{The project fund}
\label{s:fund}
Each project has an associated account $\field{P}{account}$ called the
\emph{fund}. To use that account to fund maintenance or other projects,
the \textsf{fund} transaction is used:
\[
    \tx{fund}{\field{P}{id}, \field{A}{id}, v}
\]

\spec{
    \inputs
    \begin{itemize}
        \item $\field{P}{id}$ is the id of the project that should initiate
            the value transfer.
        \item $\field{A}{id}$ is the id of the account that should receive
            the value.
        \item $v$ is the value being transfered.
    \end{itemize}
    \validation
    \begin{itemize}
        \item $\field{p}{bal} \geq v$, where $p = \field{P}{account}$,
        % TODO: Update this if we have deposits.
        \item $\field{P}{contract}(\txmsg{fund}{\field{P}{id}, \field{A}{id}, v}) \equiv \top$.
    \end{itemize}
    \outputs
    \begin{itemize}
        \item $\field{A}{bal'} = \field{A}{bal} + v$
    \end{itemize}}

\section{Identity}

\subsection{Identifying as a contributor}
The act of identifying yourself as a contributor, by linking a public key used
to sign project contributions, to an account in the registry:
\[
    \tx{identify}{\field{I}{pk}, \field{I}{proof}}
\]

\spec{
    \inputs
    \begin{itemize}
        \item $\field{I}{pk}$ is the public key that is to be associated with
            the \emph{origin} account $\alpha$ if this transaction succeeds.
        \item $\field{I}{proof}$ is a proof verifying that the transaction
            author owns $\field{I}{pk}$.
    \end{itemize}
    \validation
    \begin{itemize}
        \item $\field{I}{pk}$ is not already associated with an account,
        \item $\field{I}{proof}$ is $\alpha_{id}$ signed by the secret key $sk$ that
            $\field{I}{pk}$ was derived from. In other words,
            \[
                \field{I}{proof} = \mathsf{encrypt(hash(\alpha_{id}), \mathit{sk})}
            \]
            which is valid if
            \[
                \mathsf{decrypt}(\field{I}{proof}, \field{I}{pk}) \equiv
                    \mathsf{hash(\alpha_{id})}
            \]
    \end{itemize}}

\subsection{Forgetting identities}
When a public key associated with the \textsf{identify} transaction is lost or no
longer used, the following transaction will `forget' the association:
\[
    \tx{forget}{\field{I}{pk}}
\]

\newpage
\appendix

% Don't show appendix in TOC.
\addtocontents{toc}{\protect\setcounter{tocdepth}{0}}

\section{Alternative contribution model}
It's possible to drastically reduce the size of checkpoints by merkleizing the commit
history and making contribution claiming an explicit action on the part of
the contributor.

First, we change the checkpoint transaction's $\field{K}{contribs}$ argument from a \emph{list}
of commits to a single \emph{hash}. This hash is the Merkle root of the commit list
that was $\field{K}{contribs}$, with $\tuple{\field{C}{hash}, \field{C}{author}, \field{C}{sig}}$ at each leaf:
% TODO: Is `C_{author}` actually required?
\[
    \field{K}{contribs} = \operatorname{\mathsf{hash}}(
        [\tuple{\field{C}{hash}, \field{C}{author}, \field{C}{sig}}]
    ),
\]
where \textsf{hash} is a Merkle hash function. $\field{K}{contribs}$ is
now constant-sized. Note that in this alternative model, $\field{C}{parent}$ is
no longer used, since it can't be verified.

Then, we ask contributors to claim contributions if they wish to receive rewards, with
a new \textsf{claim-contribution} transaction:
\[
    \tx{claim-contribution}{\field{P}{id}, C_1 \dots C_n, \field{C}{proof}}
\]

\spec{
    \inputs
    \begin{itemize}
        \item $\field{P}{id}$ is the id of the project $P$ to claim the
            contributions on.
        \item $C_1 \dots C_n$ is the list of commits to claim, where $C =
            \tuple{\field{C}{hash}, \field{C}{sig}}$, and $\field{C}{author} =
            \alpha$, the author of the transaction. Note that $\field{C}{author}$
            is not actually included in the transaction, since it is assumed to always
            be $\alpha$.
        \item $\field{C}{proof}$ is the Merkle proof asserting that $C_1 \dots C_n$
            are included in the $\field{K}{contribs}$ set. It is expected that
            $\field{C}{proof}$ can use structural sharing to compress the Merkle paths
            such that the transaction does not grow linearly in $n$. Formally, each
            $\field{C}{proof}$ is an ordered set $\{\pi_1 \dots \pi_n\}$ of Merkle
            paths associated with $C_1 \dots C_n$.
    \end{itemize}
    \validation
    \begin{itemize}
        \item $\field{C}{proof}$ is valid if each path $\pi$ hashes to the same
            root hash $r$, and there is a checkpoint $K$ in $\field{P}{k}$'s
            ancestry such that $\field{K}{contribs} = r$.
        \item For each $C \in C_1 \dots C_n$, $\field{C}{sig}$ is a signature by the
            author $\alpha$.
        \item $C_1 \dots C_n$ have never been claimed before on $P$. Note that if
            two projects have a shared ancestry, it's possible to claim
            contributions once for \emph{each} project.
        % TODO: Explain how the proof is constructed?
    \end{itemize}
    \outputs
    \begin{itemize}
        \item $\alpha_{bal'} = \alpha_{bal} + v$, where $\alpha_{bal'}$ is the
            author's new account balance, and $v = \field{P}{contract}(t)$ where
            $t$ is the \textsf{claim-contribution} transaction.
    \end{itemize}
}

\section{Projects}

\subsection{Project proposal}
The act of proposing a project for registration under a unique name and domain:
\[
    \tx{propose-project}{\field{P}{id}, \field{P}{k}, \field{P}{contract},
        \field{P}{proof}, \field{P}{meta}}
\]
This transaction requires a deposit $\deposit{propose-project}$.

\spec{
\inputs
\begin{itemize}
    \item $\field{P}{id} = \tuple{\field{P}{name}, \field{P}{org}}$
    \item $\field{P}{name}$ is the unique name being requested, where
    \item $\field{P}{domain}$ is the domain under which $\field{P}{name}$ is
        being registered, which together form the unique identifier
        $\field{P}{id}$,
    \item $\field{P}{k}$ is the id of the initial \emph{checkpoint} associated
        with this project, formally $k_0$. This checkpoint must always remain
        in the project ancestry,
    \item $\field{P}{contract}$ is the initial project contract that includes
        the initial permission set around the project,
    \item $\field{P}{proof}$ is a byte array of up to 4096 bytes supplied to
        prove the legitimacy of this project proposal---it is verified
        \emph{off-registry} during project approval (See \S\ref{s:accepting-projects}),
    \item $\field{P}{meta}$ is a dictionary of metadata to be associated with
        $P$. For example, identities on other platforms. Note that once submitted,
        the metadata is immutable.
\end{itemize}
\validation
\begin{itemize}
    \item $\field{P}{name}$ must be unique, \ie not currently registered under
        $\field{P}{domain}$, between $1$ and $32$ characters long, and valid
        \textsf{UTF-8}---note that it is perfectly valid for multiple proposals
        to co-exist with the same $\field{P}{id}$.
    \item $\field{P}{domain}$ must be an existing domain,
    \item $\field{P}{proof}$'s size is $\leq 4096$ bytes,
    \item $\field{P}{k}$ must represent an existing checkpoint,
    \item $\alpha_{bal} \geq \deposit{propose-project}$.
\end{itemize}
\outputs
\begin{itemize}
    \item $P \in \pending$, where $\pending$ is the set of proposals that
        are in a ``pending'' state, waiting to be accepted or rejected (See
        \S\ref{s:accepting-projects}). Note that $\pending$ and $\projects$
        are disjoint,
    \item $\alpha_{bal'} = \alpha_{bal} - \deposit{propose-project}$.
\end{itemize}}

\subsection{Accepting and rejecting a project}
\label{s:accepting-projects}
The act of accepting or rejecting a project being registered:
\[
    \tx{accept-project}{\field{t}{hash}}
\]
or
\[
    \tx{reject-project}{\field{t}{hash}}
\]

\inputs
\begin{itemize}
    \item $\field{t}{hash}$ is the \emph{transaction hash} of the
        \textsf{propose-project} transaction $t$ of a project $P$
        being accepted or rejected.
\end{itemize}
\validation
\begin{itemize}
    \item The transaction \emph{origin} is a member of $\root$,
    \item $\field{t}{hash}$ must be the hash of an existing transaction of
        type \textsf{propose-project}. In other words, $P \in \pending$,
        where $P$ is the project being registered.
\end{itemize}

\bigskip
\noindent For \textsf{accept-project},
\newline\outputs
\begin{itemize}
    \item $P \in \projects$
    \item $P \notin \pending$
    \item $\field{P}{account} = \tuple{\field{A}{id}, 0, 0}$
\end{itemize}
\bigskip
\noindent For \textsf{reject-project},
\newline\outputs
\begin{itemize}
    \item $P \notin \projects$
    \item $P \notin \pending$
\end{itemize}

\subsection{Project proposal withdrawal}
It's possible to withdraw a project from proposal if it hasn't been
accepted or rejected yet with:
\[
    \tx{withdraw-project}{\field{t}{hash}}.
\]
This in turn returns the deposit made on \textsf{propose-project}.

\spec{
    \inputs
    \begin{itemize}
        \item $\field{t}{hash}$ is the transaction hash of the \textsf{propose-project}
            transaction $t$ for some project $P$ that should be withdrawn.
    \end{itemize}
    \validation
    \begin{itemize}
        \item $P \in \pending$
    \end{itemize}
    \outputs
    \begin{itemize}
        \item $P \notin \pending$
        \item $\alpha_{bal'} = \alpha_{bal} + \deposit{propose-project}$.
    \end{itemize}
}

\subsection{Project Tokens}
Projects are able to issue non-fungible tokens (NFTs) that can be used to
access certain services or features around the project, such as support,
licenses, issue and feature prioritization, or access to premium content.

With NFTs, the registry functions as a decentralized ``licensing server'',
which grants a token holder access to certain services. Acquiring NFTs
is done by transfer from another holder, or purchase directly from the
project.

A token is formally represented as:
\[
    \theta = \tuple{\field{\theta}{id}, \field{\theta}{proj}, \field{\theta}{tag},
                    \field{\theta}{price}}
\]
where $\field{\theta}{id}$ uniquely identifies a token unit. The set of
all tokens is $\Theta$.

\subsection{Token issuance}
\[
    \tx{issue-tokens}{\field{\theta}{proj}, \field{\theta}{tag}, \field{\theta}{price}, n}
\]

\spec{
    \inputs
    \begin{itemize}
        \item $\field{\theta}{proj}$ is the id of the project $P$ issuing the token,
        \item $\field{\theta}{tag}$ is a project-scoped tag to be associated with the issued tokens,
        \item $\field{\theta}{price}$ is the price for one token, in the native currency,
        \item $n$ is the number of tokens to issue with that tag and price.
    \end{itemize}
    \validation
    \begin{itemize}
        \item $P \in \projects$,
        \item $\field{\theta}{tag} \in \bin_{32}$,
        \item $\field{\theta}{price} \in \nat_{\geq 0}$,
        \item $n \geq 1$,
    \end{itemize}
    \outputs
    \begin{itemize}
        \item $\Theta' = \Theta \cup \theta_{0} \ldots \theta_{n}$
    \end{itemize}}
Note that $\field{\theta}{tag}$ can be re-used to re-issue tokens with the same tag (but a potentially different price) multiple times. The tag is an arbitrary value meant to represent
the type of NFT being purchased.

\subsection{Token revocation}
\[
    \tx{revoke-tokens}{\field{\theta}{id_0} \ldots \field{\theta}{id_n}}
\]

\subsection{Token purchase}
\[
    \tx{purchase-token}{\field{\theta}{id}}
\]

\subsection{Token transfer}
\[
    \tx{transfer-token}{\field{\theta}{id}, \field{A}{id}}
\]

\section{Root accounts}
Some accounts are considered \emph{privileged}. These `root' accounts,
formally $\root \subset \accounts$ are authorized to conduct certain transactions
that are only valid when originating from these accounts.

The set of accounts in $\root$ is defined at \emph{genesis}, and may not be
further modified in the initial protocol.

\section{Names}

\subsection{Registering a domain}
The act of registering a top-level domain:
\[
    \tx{register-domain}{domain}
\]
\inputs
\begin{itemize}
    \item $domain$ is the unique domain being registered.
\end{itemize}
\validation
\begin{itemize}
    \item The transaction origin $\alpha$ is a member of $\root$,
    \item $domain$ must be available for registration, between $1$ and $32$
        characters long, and valid \textsf{UTF-8}.
\end{itemize}
For example,
\[
    \tx{register-domain}{\mathtt{crates}}
\]

\subsection{Distribution contracts}
A distribution contract can be described as a function:
\[
    f : \txs \to \mathcal{O}
\]
where $\txs$ is any known transaction, and $\mathcal{O}$ is the
function's codomain described by:
\[
    \{\top, \bot\} \cup [\distribution] \cup \nat.
\]
An output in $\{\top, \bot\}$ is reserved for \emph{boolean} contracts,
where $\top$ signifies a transaction $t \in \txs$ is
\emph{authorized} to execute by the contract, and $\bot$ means it
is \emph{unauthorized}.  Note that a transaction can be verified
and included in the transaction ledger $\txs$ yet still be
unauthorized to run.

An output in $[\distribution]$ is reserved for transactions
that should trigger a distribution of funds to one or more
contract-specified beneficiaries, where $\field{A}{id}$ is the
account of one of the beneficiaries, and $\nat$ is the value to
transfer.

An output in $\nat$ is reserved for transactions that should
trigger a transfer of funds to an a priori \emph{known} beneficiary
that shouldn't be determined by the contract. For example, the transaction
origin.
