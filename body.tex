\section{Transactions}
\label{s:transactions}

All transactions on the registry take the form $\tx{transaction}{arg_1, \ldots,
arg_n}$, where $arg_1, \ldots arg_n$ are the \emph{inputs} and $\sigma$ is the
\textsf{EdDSA} signature of the author of the transaction. Transactions always
have an \emph{author} and an \emph{origin} (formally $\alpha$), which is the
author's account.

Transactions can be uniquely identified by their \emph{hash}.

\section{Accounts}
An account $A$ is a tuple:
\[
    A = \tuple{\field{A}{id}, \field{A}{nonce}, \field{A}{bal}}
\]

\spec{
    \definition
    \begin{itemize}
        \item $\field{A}{id}$ is the unique account identifier obtained by hashing
            the account owner's public key,
        \item $\field{A}{nonce}$ is a number which starts at $0$ and is incremented
            every time a transaction originates from this account.
        \item $\field{A}{bal}$ is the account's balance in the smallest denomination,
            and $\field{A}{bal} \in \nat_{\geq 0}$.
    \end{itemize}}
The set of all accounts is $\accounts$. Accounts are never created or destroyed,
rather, if they have never been used to transact, they have an initial state of:
\[
    A = \tuple{\field{A}{id}, 0, 0}
\]
Hence, for all valid account ids, there exists an account with that id. In other
words, $\forall a \in \field{A}{id} (A \in \accounts)$.

Note that accounts can \emph{never be removed}, since that would violate the
invariant that nonces are only ever incremented, and removing an account is
equivalent to setting $\field{A}{nonce}$ and $\field{A}{bal}$ to $0$.

\subsection{Root accounts}
Some accounts are considered \emph{privileged}. These `root' accounts,
formally $\root \subset \accounts$ are authorized to conduct certain transactions
that are only valid when originating from these accounts.

The set of accounts in $\root$ is defined at \emph{genesis}, and may not be
further modified in the initial protocol.

\subsection{Transferring value}
The act of transferring coins between two accounts:
\[
    \tx{transfer}{\field{A}{id}, v}
\]
which will transfer value from the transaction origin $\alpha$ to account $A$.

\spec{
    \inputs
    \begin{itemize}
        \item $\field{A}{id}$ is the account id of the \emph{receiver} of the transfer,
        \item $v$ is the value or `balance' to transfer from the origin to the receiver,
            in the smallest denomination.
    \end{itemize}

    \validation
    \begin{itemize}
        \item The transfer balance is positive, or $v \geq 1$,
        \item The origin's balance minus any transaction fee is $\geq v$.
    \end{itemize}
    \outputs
    \begin{itemize}
        \item $v$ is debited from the origin and credited to $A$.
    \end{itemize}
}

\section{Projects}
A project $P$ is a tuple:
\[
    P = \tuple{\field{P}{id}, \field{P}{k}, \field{P}{account},
        \field{P}{contract}, \field{P}{proof}, \field{P}{meta}}
\]

\spec{
    \definition
    \begin{itemize}
        \item $\field{P}{id}$ is the unique project identifier, defined as
            $\tuple{\field{P}{name}, \field{P}{domain}}$,
        \item $\field{P}{k}$ is the current project \emph{checkpoint} (See \S\ref{s:checkpoint}),
        \item $\field{P}{account}$ is the project account or \emph{fund},
        \item $\field{P}{contract}$ is the project contract, which governs
            permissions around the project, as well as its fund. It can be
            invoked as a function which takes a transaction $t$ as input, and
            returns an output in $\{\top, \bot\}$, or:
            \[
                \field{P}{contract}(t) \to \{\top, \bot\}.
            \]
            An output of $\top$ signifies \emph{authorized}, while $\bot$ is for
            \emph{unauthorized}.
        \item $\field{P}{proof}$ is the proof that was supplied during registration,
            verifying the owner's authority over the project,
        \item $\field{P}{meta}$ is a dictionary of additional metadata to associate
            with the project, for example the \radicle{} \emph{project id}.
            $\field{P}{meta}$ is \emph{immutable} once defined.
    \end{itemize}}
Projects are created with the \textsf{register-project} transaction.  The set
of all projects is $\projects$.

\subsection{Project registration}
The act of registering a project under a unique name and domain:
\[
    \tx{register-project}{\field{P}{id}, \field{P}{k}, \field{P}{proof}, \field{P}{meta}}
\]

\inputs
\begin{itemize}
    \item $\field{P}{id} = \tuple{\field{P}{name}, \field{P}{domain}}$
    \item $\field{P}{name}$ is the unique name being requested, where
    \item $\field{P}{domain}$ is the domain under which $\field{P}{name}$ is
        being registered, which together form the unique identifier
        $\field{P}{id}$,
    \item $\field{P}{k}$ is the id of the initial \emph{checkpoint} associated
        with this project, formally $k_0$. This checkpoint must always remain
        in the project ancestry.
\end{itemize}
\validation
\begin{itemize}
    \item $\field{P}{name}$ must be unique, \ie not currently registered under
        $\field{P}{domain}$, between $1$ and $32$ characters long, and valid
        \textsf{UTF-8},
    \item $\field{P}{domain}$ must be an existing domain,
    \item $\field{P}{k}$ must represent an existing checkpoint.
\end{itemize}

\subsection{Accepting and rejecting a project}
The act of accepting or rejecting a project being registered:
\[
    \tx{accept-project}{\field{t}{hash}}
\]
or
\[
    \tx{reject-project}{\field{t}{hash}}
\]

\inputs
\begin{itemize}
    \item $\field{t}{hash}$ is the \emph{transaction hash} of the
        \textsf{register-project} transaction $t$ of a project
        being accepted or rejected.
\end{itemize}
\validation
\begin{itemize}
    \item The transaction \emph{origin} is a member of $\root$,
    \item $\field{t}{hash}$ must be the hash of an existing transaction of
        type \textsf{register-project},
    \item $t$ must not have been previously accepted or rejected, in other
        words there can be at most one \textsf{accept-project} or
        \textsf{reject-project} for each $t$.
\end{itemize}

\bigskip
\noindent For \textsf{accept-project},
\newline\outputs
\begin{itemize}
    \item $P \in \projects$
    \item $\field{P}{account} = \tuple{\field{A}{id}, 0, 0}$
\end{itemize}
\bigskip
\noindent For \textsf{reject-project},
\newline\outputs
\begin{itemize}
    \item $P \notin \projects$
\end{itemize}

\subsection{Checkpointing}
\label{s:checkpoint}
The act of notarizing a project's state and updating the network graph:
\[
    \tx{checkpoint}{\field{K}{parent}, \field{K}{hash}, \field{K}{version},
    \field{K}{contribs}, \field{K}{deps}}
\]
Checkpoints form a chain going from the latest checkpoint to the first.

\inputs
\begin{itemize}
    \item $\field{K}{parent}$ is the \emph{id} of the previous or `parent' checkpoint,
    \item $\field{K}{hash}$ is the new \emph{hash} of the project state,
    \item $\field{K}{version}$ is the new \emph{version} of the project,
    \item $\field{K}{contribs}$ is the list of contributions since $\field{K}{parent}$,
    \item $\field{K}{deps}$ is the list of dependency updates since the $\field{K}{parent}$.
\end{itemize}
\validation
\begin{itemize}
    \item{$\field{K}{parent}$ refers to an existing checkpoint in the registry,
        or is $\varnothing$.}
    \item{$\field{K}{hash}$ is a valid hash that hasn't been used in a parent
        checkpoint.}
    \item{$\field{K}{version}$ is a string between $1$ and $32$ bytes long that
        hasn't been used in a previous project checkpoint.}
    \item{$\field{K}{contribs}$ is a valid contribution list (See \S
        \ref{s:checkpoint-contribs}).}
    \item{$\field{K}{deps}$ is a valid dependency update list (See \S
        \ref{s:checkpoint-deps}).}
\end{itemize}

\subsubsection{Contributions}
\label{s:checkpoint-contribs}
The list $\field{K}{contribs}$ supplied to the $\mathsf{checkpoint}$
transaction is of the form:

\[
    \field{K}{contribs} = [\tuple{\field{C}{parent}, \field{C}{hash},
    \field{C}{author}, \field{C}{sig}}],
\]

\spec{
    \definition
    \begin{itemize}
        \item $\field{C}{parent}$ is the hash of the parent contribution, or
            $\varnothing$ if this is the first contribution of the first checkpoint
            of the project.
        \item $\field{C}{hash}$ is the hash of the corresponding commit,
        \item $\field{C}{author}$ is the public signing key of the commit referred
            to by $C$,
        \item $\field{C}{sig}$ is the author's \textsf{GPG} signature.
    \end{itemize}
    \validation
    \begin{itemize}
        \item $\field{C}{parent}$ is a valid \textsf{SHA-1} hash or
            $\varnothing$ if this is the first contribution. Note that if $C$
            is $\field{K}{contribs}$'s first item, and $C'$ is the \emph{last}
            item of the \emph{parent} checkpoint's contributions list, then
            $\field{C'}{hash}$ and $\field{C}{parent}$ must be equal, such that
            no gaps between contributions exist.

        \item $\field{C}{hash}$ is a valid \textsf{SHA-1} hash,
        \item $\field{C}{author}$ is the creator of $\field{C}{sig}$,
        \item $\field{C}{sig}$ is a valid signature of $\field{C}{hash}$.
    \end{itemize}
}
Because all changes to a project's source code are described in checkpoints, it
is possible to reconstruct a full hash-linked list of contributions for the
entire project. When cross-referenced with the project's repository, this
constitutes a complete historical record of who authored what code. This
ensures the project history is auditable and tamper-proof, while providing
fundamental information to for the network graph $\netgraph$. Note that only
contribution \emph{metadata} is stored on-chain.

\subsubsection{Dependency updates}
\label{s:checkpoint-deps}
Conceptually, a project $P$ depends on another project $P'$ if it is an
``input'' to $P$ in some way: $P$ references $P'$ or parts of $P'$ in its
source code, or $P'$ is a build/test dependency.

The dependency update list $\field{P}{deps}$ is a list of \emph{dependency
  updates}, one of:
\[
    \begin{cases}
        \depend(\field{P'}{id}, \field{P'}{version}) \\
        \undepend(\field{P'}{id}, \field{P'}{version})
    \end{cases}
\]
which refer to the project $P'$ at a specific version $\field{P'}{version}$.
The $\depend$ update adds a new dependency while the $\undepend$ update removes
a dependency. The updates are processed in order with $\depend$ only being
valid if it adds a dependency that the project does not already have and
$\undepend$ only being valid for current dependencies. The checkpoint is
invalid if the update list contains duplicates.

% NOTE: Should the checkpoints include the hash of the project being depended
% on? This would add another check for projects registering.

\bigskip
\validation
\begin{itemize}
    \item $\field{P'}{id}$ must be a valid project id, but \emph{does not}
        have to refer to an existing id in the registry. This allows dependent
        projects to checkpoint dependencies that have not yet been registered.
    \item $\field{P'}{version}$ must be a valid version string, but \emph{does not}
        have to refer to an existing version of $P'$. This allows dependent projects
        to checkpoint before their dependencies.
\end{itemize}
As a project maintainer, adding a dependency signals a variety of things
depending of the nature of the project:
\begin{itemize}
\item They have verified that $P$ indeed depends on this specific
  version of $P'$.
\item That $P'$ is suitable as a dependency for $P$, \eg{} if $P$ has
  very high security requirements, that $P'$ fulfills these.
\end{itemize}
\noindent Since contributions to a project carry additional
weight---potentially increasing a project's rank---there is an incentive
for maintainers to checkpoint their projects regularly.  Similarly, adding
dependencies may increase connectivity in the network graph, which may in turn
indirectly improve a project's rank.

\subsection{Setting the project checkpoint}
The act of updating the project to point to a new checkpoint:
\[
    \tx{set-checkpoint}{\field{P}{id}, k'}
\]
which updates $\field{P}{checkpoint}$ from $k$ to $k'$.

\spec{
    \inputs
    \begin{itemize}
        \item $\field{P}{id}$ is the id of the project being updated,
        \item $k'$ is the id of the checkpoint the project
            should be associated to.
    \end{itemize}
    \validation
    \begin{itemize}
        \item $\field{P}{id}$ refers to a project that has been accepted
            in the registry,
        \item $k'$ is a checkpoint which has the original project checkpoint
            $k_0$ in its ancestry.
        \item $\field{P}{contract}(\tx{set-checkpoint}{\field{P}{id}, k'}) \equiv \top$
    \end{itemize}
    \outputs
    \begin{itemize}
        \item $\field{P}{k} = k'$
    \end{itemize}}

\subsection{Updating the project contract}
The act of updating a project's contract:
\[
    \tx{set-contract}{\field{P}{id}, c}
\]

\spec{
    \inputs
    \begin{itemize}
        \item $\field{P}{id}$ is the id of the project,
        \item $c$ is the new contract.
    \end{itemize}
    \validation
    \begin{itemize}
        \item $P \in \projects$,
        \item $\field{P}{contract}(\tx{set-contract}{\field{P}{id}, c}) \equiv \top$
    \end{itemize}
    \outputs
    \begin{itemize}
        \item $\field{P}{contract} = c$
    \end{itemize}}

\section{Names}

\subsection{Registering a domain}
The act of registering a top-level domain:
\[
    \tx{register-domain}{domain}
\]
\inputs
\begin{itemize}
    \item $domain$ is the unique domain being registered.
\end{itemize}
\validation
\begin{itemize}
    \item The transaction origin $\alpha$ is a member of $\root$,
    \item $domain$ must be available for registration, between $1$ and $32$
        characters long, and valid \textsf{UTF-8}.
\end{itemize}
For example,
\[
    \tx{register-domain}{\mathtt{crates}}
\]

\section{Identity}

\subsection{Identifying as a contributor}
The act of identifying yourself as a contributor, by linking a public key used
to sign project contributions, to an account in the registry:
\[
    \tx{identify}{\field{I}{pk}, \field{I}{proof}}
\]

\spec{
    \inputs
    \begin{itemize}
        \item $\field{I}{pk}$ is the public key that is to be associated with
            the \emph{origin} account $\alpha$ if this transaction succeeds.
        \item $\field{I}{proof}$ is a proof verifying that the transaction
            author owns $\field{I}{pk}$.
    \end{itemize}
    \validation
    \begin{itemize}
        \item $\field{I}{pk}$ is not already associated with an account,
        \item $\field{I}{proof}$ is $\alpha_{id}$ signed by the secret key $sk$ that
            $\field{I}{pk}$ was derived from. In other words,
            \[
                \field{I}{proof} = \mathsf{encrypt(hash(\alpha_{id}), \mathit{sk})}
            \]
            which is valid if
            \[
                \mathsf{decrypt}(\field{I}{proof}, \field{I}{pk}) \equiv
                    \mathsf{hash(\alpha_{id})}
            \]
    \end{itemize}}

\subsection{Forgetting identities}
When a public key associated with the \textsf{identify} transaction is lost or no
longer used, the following transaction will `forget' the association:
\[
    \tx{forget}{\field{I}{pk}}
\]
