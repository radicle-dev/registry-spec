\section{Transactions}
\label{s:transactions}

All transactions on the registry take the form $\tx{transaction}{arg_1, \ldots,
arg_n}$, where $arg_1, \ldots arg_n$ are the \emph{inputs} and $\sigma$ is the
\textsf{EdDSA} signature of the author of the transaction. Transactions
always have an \emph{author} and an \emph{origin}, which is the account of
their author.

Transactions can be uniquely identified by their \emph{hash}.

\subsection{Transfer}
The act of transferring coins between two accounts:
\[
    \tx{transfer}{\field{A}{id}, v}
\]
which will transfer value from the author's account to account $A$.

\inputs
\begin{itemize}
    \item $\field{A}{id}$ is the account id of the \emph{receiver} of the transfer,
    \item $v$ is the value or `balance' to transfer, in the smallest denomination.
\end{itemize}
\validation
\begin{itemize}
    \item $\field{A}{id}$ is a valid account id,
    \item $A$ is an account with a balance $\geq 0$,
    \item $v$ is the value or `balance' to transfer, in the smallest denomination.
    \item The author's account balance minus any transaction fee is $\geq v$.
\end{itemize}
\outputs
\begin{itemize}
    \item $v$ is debited from the author's account and credited to $A$.
\end{itemize}

\subsection{Checkpoint}
\label{s:checkpoint}
The act of notarizing project state and updating the network graph:
\[
    \tx{checkpoint}{\field{P}{id}, \field{P}{hash}, \field{P}{version}, \field{P}{contribs}, \field{P}{deps}}
\]
\inputs
\begin{itemize}
    \item $\field{P}{id}$ is the \emph{id} of the project being checkpointed,
    \item $\field{P}{hash}$ is the new \emph{hash} of the project state,
    \item $\field{P}{version}$ is the new \emph{version} of the project,
    \item $\field{P}{contribs}$ is the list of contributions since the last checkpoint,
    \item $\field{P}{deps}$ is the list of dependency updates since the last checkpoint.
\end{itemize}
\validation
\begin{itemize}
    \item{$\field{P}{id}$ refers to an existing project in the registry.}
    \item{$\field{P}{hash}$ is a valid hash that hasn't been used in a previous
        project checkpoint.}
    \item{$\field{P}{version}$ is a string between $1$ and $32$ bytes long that
        hasn't been used in a previous project checkpoint.}
    \item{$\field{P}{contribs}$ is a valid contribution list (See \S
        \ref{s:checkpoint-contribs}).}
    \item{$\field{P}{deps}$ is a valid dependency update list (See \S
        \ref{s:checkpoint-deps}).}
\end{itemize}

\subsubsection{Contributions}
The list $\field{P}{contribs}$ supplied to the $\mathsf{checkpoint}$
transaction is of the form:

\[
    [\tuple{\field{C}{prev}, \field{C}{hash}, \field{C}{author}, \field{C}{sig}, \field{C}{signoff}}]
\]
where:
\begin{itemize}
    \item $\field{C}{prev}$ is the hash of the previous contribution, or
      $\varnothing$ if this is the first contribution to the whole
      project. Note that $\field{P}{contribs}$'s first item must be linked to the last
      contribution in the project's \emph{previous} checkpoint such that no
      gaps between contributions exist.
    \item $\field{C}{hash}$ is the hash of the corresponding commit,
    \item $\field{C}{author}$ is the author of the contribution,
    \item $\field{C}{sig}$ is the author's signature,
    \item $\field{C}{signoff}$ is the \emph{signoff key}, which must be $\in P_K$.
\end{itemize}

A contribution must be signed by the signoff key. This signals
the contribution has been reviewed and verified by the maintainer. The
checkpoint itself must also be signed by a key in $P_K$.

Because all changes to a project's source code are described in checkpoints, it
is possible to reconstruct a full hash-linked list of contributions for the
entire project. When cross-referenced with the project's repository, this
constitutes a complete historical record of who authored what code, and which
maintainer signed off on the contribution. This ensures the project history is
auditable and tamper-proof, while providing fundamental information to
\osrank{}. Note that only contribution \emph{metadata} is stored
on-chain.

\subsubsection{Dependency updates}
\label{s:checkpoint-deps}
Conceptually, a project $P$ depends on another project $P'$ if it is an
``input'' to $P$ in some way: $P$ references $P'$ or parts of $P'$ in its
source code, or $P'$ is a build/test dependency.

The dependency update list $\field{P}{deps}$ is a list of \emph{dependency
  updates}, one of:
\[
    \begin{cases}
        \depend(\field{P'}{name}, \field{P'}{version}) \\
        \undepend(\field{P'}{name}, \field{P'}{version})
    \end{cases}
\]
which refer to the project $P'$ at a specific version $\field{P'}{version}$.
The $\depend$ update adds a new dependency while the $\undepend$ update removes
a dependency. The updates are processed in order with $\depend$ only being
valid if it adds a dependency that the project does not already have and
$\undepend$ only being valid for current dependencies. The checkpoint is
invalid if the update list contains duplicates.

% NOTE: Should the checkpoints include the hash of the project being depended
% on? This would add another check for projects registering.

\bigskip
\validation
\begin{itemize}
    \item $\field{P'}{name}$ must be a valid project name, but \emph{does not}
        have to refer to an existing name in the registry. This allows dependent
        projects to checkpoint dependencies that have not yet been registered.
    \item $\field{P'}{version}$ must be a valid version string, but \emph{does not}
        have to refer to an existing version of $P'$. This allows dependent projects
        to checkpoint before their dependencies.
\end{itemize}

As a project maintainer, adding a dependency signals a variety of things
depending of the nature of the project:
\begin{itemize}
\item They have verified that $P$ indeed depends on this specific
  version of $P'$.
\item That $P'$ is suitable as a dependency for $P$, \eg{} if $P$ has
  very high security requirements, that $P'$ fulfills these.
\end{itemize}
\noindent Since contributions to a project carry additional
weight---potentially increasing a project's \osrank{}---there is an incentive
for maintainers to checkpoint their projects regularly.  Similarly, adding
dependencies may increase connectivity in the network graph, which may in turn
indirectly improve a project's \osrank{}.

\subsection{Register domain}
The act of registering a top-level domain:
\[
    \tx{register\hbox{-}domain}{domain}
\]
\inputs
\begin{itemize}
    \item $domain$ is the unique domain being registered.
\end{itemize}
\validation
\begin{itemize}
    \item $domain$ must be available for registration, between $1$ and $32$
        characters long, and valid UTF-8.
\end{itemize}
For example,
\[
    \tx{register\hbox{-}domain}{\mathtt{crates}}
\]

\subsection{Claim}
The act of making a request for a name under a \emph{TLD} (top-level domain):
\[
    \tx{claim}{name, domain},
\]
\inputs
\begin{itemize}
    \item $name$ is the unique name being requested,
    \item $domain$ is the domain under which the name is being requested.
\end{itemize}
\validation
\begin{itemize}
    \item $name$ must be unique, \ie not currently registered under $domain$,
        between $1$ and $32$ characters long, and valid UTF-8.
    \item $domain$ must be an existing domain.
\end{itemize}
For example,
\[
    \tx{claim}{\mathtt{criterion}, \mathtt{crates}}
\]
which will request \texttt{criterion.crates}.

\subsection{Accept/reject claim}
The act of accepting or rejecting a claim:
\[
    \tx{accept\hbox{-}claim}{\field{C}{id}}
\]
or
\[
    \tx{reject\hbox{-}claim}{\field{C}{id}}
\]
where $C$ is the claim being accepted or rejected.

\inputs
\begin{itemize}
    \item $\field{C}{id}$ is the \emph{transaction hash} of the claim.
\end{itemize}
\validation
\begin{itemize}
    \item $\field{C}{id}$ must be the hash of an existing transaction of
        type $\mathsf{claim}$.
        The claim $C$ must be in a $pending$ state, \ie it must not have
        been already accepted or rejected.
\end{itemize}

\subsection{Identify}
The act of identifying yourself as a contributor, by linking a public key used
to sign project contributions, to an account in the registry.

% Questions
% - Does a project need to register, or can it simply checkpoint?
% - What happens when a project forks? Can we share checkpoints? What is the project id then?
